\documentclass{SGGW-thesis}

\INZYNIERSKAtrue
\WZIMtrue

\title{Porównanie aplikacji frontendowych opartych na mikrofrontendach z tradycyjną architekturą monolityczną na przykładzie aplikacji do zarządzania finansami osobistymi}
\Etitle{Comparison of microfrontend applications and monolith frontend applications based on the example of expense tracker }
\author{Dawid Wijata}
\date{2022}
\album{205006}
\thesis{Praca dyplomowa na kierunku:}
\course{Informatyka}
\promotor{dr \ inż.\ Piotra Wrzeciono}
\pworkplace{Instytut Informatyki Technicznej\\Katedra Systemów Informacyjnych}
\usepackage{hyperref}

\begin{document}
\maketitle
\statementpage
\abstractpage
{Stworzenie klasy \LaTeX-owej do użytku przy pisaniu pracy dyplomowej w SGGW}
{Tematem niniejszej pracy było zaimplementowanie klasy \LaTeX-owej pozwalającej na formatowanie tekstu zgodnie z wytycznymi nałożonymi przez uczelnię. Praca zawiera dwie
główne części. Pierwsza z nich zawiera opis najważniejszych aspektów implementacji klasy. Natomiast druga część skupia się na sposobie użycia klasy przez osoby piszące prace
dyplomowe.}
{LaTeX, klasa, praca dyplomowa, implementacja, SGGW, Szkoła Główna Gospodarstwa Wiejskiego}
{Creation of the \LaTeX\ Class to be Used When Writing a Thesis at the Warsaw University of Life Sciences -- SGGW}
{The subject of this study was to implement a \LaTeX\ class that allows for text formatting according to the guidelines imposed by the University. The work consists of two
main parts. The first one describes the most important aspects of the implementation. The second part focuses on how to use the class by people writing the theses.}
{LaTeX, class, thesis, implementation, SGGW, Warsaw University of Life Sciences}


{
  % Spis treści może być złożony z pojedynczą interlinią, np. jeśli jedna linia wychodzi na następną stronę.
  % W przeciwnym razie spis treści wstawić bez powyższego rozkazu i klamry.
  \doublespacing
  \tableofcontents
}

\startchapterfromoddpage % niezależnie od długości spisu treści pierwszy rozdział zacznie się na nieparzystej stronie

\chapter{Testowy rozdział}
Pisanie pracy dyplomowej oprócz oczywistego aspektu przekazywania informacji wiąże się także z odpowiednim sformatowaniem wynikowego tekstu według zasad narzuconych przez
uczelnię. 


\begin{thebibliography}{9}
\bibitem{wymagania}
\textit{Zarządzenie nr 34 Rektora Szkoły Głównej Gospodarstwa Wiejskiego w Warszawie z dnia 01 czerwca 2016 r.\ w~ sprawie wprowadzenia ,,Wytycznych dotyczących
przygotowywania prac dyplomowych w~Szkole Głównej Gospodarstwa Wiejskiego w Warszawie''}, Załączniki 1 i~2
\url{https://www.sggw.pl/dla-studentow/informacje-formalno-prawne/dokumenty-do-pobrania}
$\rightarrow$ Praca dyplomowa (dostęp: 04.01.2017)
\bibitem{latexclass}
\textit{\LaTeXe\ for class and package writers} \url{http://www.latex-project.org/help/documentation/} $\rightarrow$ LaTeX2e for class... (dostęp 04.01.2017)
\bibitem{latexcompanion}
Frank Mittelbach and Michel Goossens with Johannes Braams, David Carlisle and Chris Rowley,
\textit{The \LaTeX\ Companion}. Second Edition,
Addison-Wesley, 2004.
\bibitem{tds}
\textit{A Directory Structure for \TeX\ Files} \url{http://tug.org/tds/tds.html} (dostęp 11.01.2017)
\end{thebibliography}

\beforelastpage

\end{document}
